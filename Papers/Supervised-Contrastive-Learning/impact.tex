\section*{Broader Impact}

This work provides a technical advancement in the field of supervised classification, which already has tremendous impact throughout industry. Whether or not they realize it, most people experience the results of this type of classifier many times a day.

As we have shown, supervised contrastive learning can improve both the accuracy and robustness of classifiers, which for most applications should strictly be an improvement. For example, an autonomous car that makes a classification error due to data distribution shift can result in catastrophic results. Thus decreasing this class of error undoubtedly promotes safety. Human driver error is a massive source of fatalities around the world, so improving the safety of autonomous cars furthers the efforts of replacing human drivers. The flip side of that progress is of course the potential for loss of employment in fields like trucking and taxi driving. Similar two-sided coins can be considered for assessing the impact of any application of classification.

An additional potential impact of our work in particular is showing the value of training with large batch sizes. Generally, large batch size training comes at the cost of substantial energy consumption, which unfortunately today requires the burning of fossil fuels, which in turn warms our planet. We are proud to say that the model training that was done in the course of this research was entirely carbon-neutral, where all power consumed was either green to start with, or offset by purchases of green energy. There is unfortunately no way to guarantee that once this research is publicly available that all practitioners of it will choose, or even have the ability to choose, to limit the environmental impact of their model training.

\section*{Acknowledgments and Disclosure of Funding}

Additional revenues related to this work: In the past 36 months, Phillip Isola has had employment at MIT, Google, and OpenAI; honorarium for lecturing at the ACDL summer school in Italy; honorarium for speaking at GIST AI Day in South Korea. P.I.'s lab at MIT has been supported by grants from Facebook, IBM, and the US Air Force; start up funding from iFlyTech via MIT; gifts from Adobe and Google; compute credit donations from Google Cloud.
